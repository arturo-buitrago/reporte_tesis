\setlang{USenglish}
\thispagestyle{plain}

\section*{Abstract}

The predicted future of wireless cellular networks promises an explosion in the number of devices connecting to every Base Station, particularly in dense urban scenarios. With the prospective advent of the Internet of Things (IoT) and Smart Cities, there arises a need to alleviate the demands these devices will place on the existing infrastructure.

Despite many proposals for the reuse of time and frequency resources through cluster-based data aggregation in LTE-A through Device-to-Device (D2D) communication, few of them have been evaluated in a realistic scenario and compared to each other fairly.

In this thesis, we develop a realistic model for the simulation of highly dense cities and the clustering of network-capable devices. We also compare different algorithms for the creation of said clusters in order to gauge their suitability.

Our results show that of the algorithms used, Single-Linkage clustering with a predefined number of clusters outperforms similar schemes, both hierarchical and otherwise in alleviating the pressure on the eNB, while dealing with interference within and without the cluster effectively.

The work here presented shows a way forward in terms of selecting clustering schemes and leaves behind a simulation environment that will prove useful in future investigations in the area of resource reuse in LTE-A.