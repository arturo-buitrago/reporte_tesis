\chapter{Conclusions and Outlook}

In this thesis we sought to contribute to prospective solutions to problems well known in our field of study: as technology moves forward towards 5G, the Internet of Things and Smart Cities, the prospective explosion in the number of devices accessing wireless networks will place an inordinate burden on our existing infrastructure. We delved deeper into the use of D2D communications as a means to reuse the resources already available within LTE-A through cluster-centric data aggregation. Recognizing a gap in the evaluation of existing proposals, we intended to fairly compare and assess a handful of clustering schemes in suitability to handle the types of scenarios which will be common in a not so distant future.

This work presents such a comparison within reason, despite some of its selective measures being less telling than was anticipated. It establishes a model that, while flawed in some aspects, deals with interference in the network, as well as the random access procedure of LTE-A effectively, and can be easily and readily expanded and amended to suit the future needs of investigation.

Our results show that Single-Linkage clustering with a predetermined number of target clusters outperforms most of its peers in all selected metrics and presents a very viable solution to the need to alleviate some of the pressure that the eNB will experience as a result of a high density of connecting devices. Despite being a very-well studied algorithm, it necessitates knowledge of the positions of all devices and presents a $O(n^2)$ computational complexity, which may pose a problem for time-constrained applications \cite{Everitt2011}. In this case, LEACH poses a very viable alternative, especially in higher densities of both UEs and CHs, where its behavior very closely resembles that of Single-Link clustering without the global knowledge or the complicated computations.

A big open question is of course, the decision to fix the arrival rate $\lambda_A$. If the data being transferred is highly correlated and includes redundancy, as is often the case with WSNs, then lower request creation rate with higher number of devices makes sense: the frequency of the information that needs to be transmitted can afford to be lower. However, our simulation scenario was thought out to have very disparate data sources, making it intrinsically different from the WSN scenario. More research is needed with a $\lambda_A$ defined at the UE and not at the cluster head, to better evaluate the negative (and even potentially positive) effect that increased interference both within and without the cluster may cause. Thankfully this sort of adjustment seems relatively simple to make and integrate into the current models for future work. 

This leads us to the successful establishment of a simulation environment for evaluation of LTE-A and especially D2D, one of the main goals of this thesis. Despite much trial and error, the work presented here will, in the mind of the author, make a positive impact in his future investigative efforts, especially in the context of already ongoing studies in the institute.

The way for future work seems relatively clear. Firstly, the definition and implementation of an arrival rate that better reflects prospective traffic in a system with a growing number of connected devices to the cluster head is of paramount importance. In this way, the impact of interference in the studied systems can be understood more profoundly. Secondly, expanding the simulation into the traffic at the eNB, both from aggregated data and normal cellular traffic will better model the actual impact that D2D has on the reuse of the available pool of resources. Finally, the incorporation of other kinds of algorithms and schemes for aggregation, for instance multi-hop or power-controlling systems will expand the repertoire of mechanisms that we can evaluate.
