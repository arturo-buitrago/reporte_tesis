\chapter{Introduction}

One of the most promising directions the field of communication networks has explored in the last years has been the Internet of Things (IoT) \cite{Shariatmadari2015a}. An explosion in the capability of everyday objects to connect and communicate with each other via wireless networks promises to mightily expand the boundaries hitherto explored by technology. It also promises to place an almost unmanageable amount of stress into the technologies and infrastructure already in place \cite{Polese2016}. 

One of the proposed approaches to ameliorate the overload of connections originating from hundreds of devices to a base station is the grouping of the signals via different algorithms into clusters, which then transmit the aggregated information in one single signal to the rest of the network \cite{Laya2014a}. This is compounded by the prospective boons that Device-to-Device (D2D) communication promises when integrated into the network \cite{6163598}. 

Although many such algorithms have been proposed, especially coming from the field of Wireless Sensor Networks (WSNs) where data aggregation is much more a matter of course \cite{Afsar2014}, there has been woefully little attention payed to the viability of such mechanisms in comparison to one another in a scenario conforming to the standards and circumstances of LTE-A and D2D communication. These kind of considerations are specially relevant when considering the prospective arrival of the IoT, the emergence of concepts such as smart grids and smart cities and the prospect of 5G as the next generation of technology that will have to deal with these issues \cite{6568922}. 

In this work we will be restricting ourselves to exactly this framework: an urban, highly dense scenario, where traffic inside clusters is observed and measured, while accounting for interference both within and without these structures. The clustering algorithms used will all be single-hop, i.e. with only one aggregation stage before connecting to the base station. The models used, when possible, will all adhere to the modern standards of LTE-A, including its procedure for random access and collision resolution. Monitoring the traffic at the base station itself, or other kinds of more complex algorithms are outside the scope of this thesis.

This thesis aims to create a coherent and realistic simulation scenario for the evaluation of traffic within an increasingly important scenario for LTE-A, that of a highly dense, highly interconnected city. This simulation's usefulness will not be restricted to this thesis but will be able to be incorporated into future work in the field, by its author or otherwise. This work also intends to quantify the adequacy of different clustering schemes in aggregating data effectively and alleviating congestion in the network. This will allow future research to better focus its efforts on algorithms that are viable. Finally, this work's contribution dovetails into the larger scope of work done in the institute at which it was written in the areas of D2D and Random Access.

The rest of the thesis will be organized as follows: First, a background section will introduce the topic and situate it within the state of research and technological standards. Next, the implementation chapter will detail the steps taken to create an appropriate simulation environment for our investigation. Finally, the results will be presented and discussed. The conclusion will recapitulate the outcomes and identify opportunities for improvement and future research.
