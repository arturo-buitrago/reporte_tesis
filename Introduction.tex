\chapter{Introduction}

One of the most promising directions the field of communication networks has explored in the last years has been the Internet of Things (IoT). An explosion in the capability of everyday objects to connect and communicate with each other via wireless networks promises to mightily expand the boundaries hitherto explored by technology. It also promises to place an almost unmanageable amount of stress into the technologies and infrastructure already in place. 

One of the proposed approaches to ameliorate the overload of connections originating from hundreds of devices to a base station is the grouping of the signals via different algorithms into clusters, which then transmit the aggregated information in one single signal to the rest of the network. 

Although many such algorithms have been proposed, especially coming from the field of Wireless Sensor Networks (WSNs) where data aggregation is much more a matter of course, there has been woefully little attention payed to the viability of such mechanisms in comparison to one another in a scenario conforming to the standards and circumstances of LTE-A and Device to Device (D2D) communication. These kind of considerations are specially relevant when considering the prospective arrival of the IoT, the emergence of concepts such as smart grids and smart cities and the prospect of 5G as the next generation of technology that will have to deal with these issues. 

This thesis aims to do just that: present a coherent and realistic simulation scenario for the clustering of devices within LTE-A, with special attention payed to the interference caused by the simultaneous transmission of information, both within the clusters and between other clusters. This will allow a fair comparison of existing clustering schemes and the degree to which they effectively alleviate congestion in the network.

The main part of the thesis will first present the background of the topic, explaining the difficulties arising from an increase in network-capable devices to the limits posed by the Random Access Procedure in place at the moment. The trade-offs of clustering will be explored, along with an enumeration of some of the most importants algorithms taken into account in this work. This chapter will also give an overview of the factors that need to be contemplated during the simulation of the network, along with the motivation for the choice of the simulation environment. Additionaly, justification will be given for the metrics used for evaluating the results of the clustering algorithms.

Next, the implementations of said simulations will be explored. This chapter will delve deeper into the details of bringing the theoretical models into the code used, with a more in-depth discussion of the decisions taken when finalizing the constraints under which the tests were run. The results that they yield will then be discussed and evaluated in the subsequent chapter, with especial attention payed to a direct comparison of the performance of different clustering algorithms. Finally, a discussion of the results will identify the ones most promising for future research and further development.
