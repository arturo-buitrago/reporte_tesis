\chapter{Notation und Abkürzungen}
This chapter contains tables where all abbreviations and other notations like mathematical
placeholders used in the thesis are listed.
\begin{table}[h]
\begin{tabular}{ll}
BS & Base Station\\
BER & Bit-Error-Rate\\
CH & Cluster Head\\
D2D & Device-to-Device\\
eNodeB & Evolved Node B or E-UTRAN Node B\\
IMT-A & International Mobile Telecommunications-Advanced\\
IoT & Internet of Things\\
ITU & International Telecommunication Union\\
LOS & Line-of-Sight\\
LTE & Long Term Evolution\\
MCS & Modulation and Coding Scheme\\
METIS & Mobile and wireless communications Enablers for the Twenty-twenty Information Society \\
MTC & Machine Type Communication\\
NLOS & Non-Line-of-Sight\\
OFDM & Orthogonal Frequency Division Multiplexing\\
O2I & Outside-to-Inside\\
O2O & Outisde-to-Outside\\
PDCCH & Physical Downlink Control Channel\\
PDSCH & Physical Downlink Shared Channel\\
PPP & Poisson Point Process\\
PRB & Physical Resource Block\\
PUCCH & Physical Uplink Control Channel\\
PUSCH & Physical Uplink Shared Channel\\
QoS & Quality of Service\\
RA & Random Access\\
RACH & Random Access Channel\\
SINR & Signal-to-Interference-and-Noise-Ratio\\
SNR & Signal-to-Noise-Ratio\\ 
SR & Scheduling Request\\
UE & User Equipment\\
WINNER+ & Wireless World Initiative New Radio+ \\
WSN & Wireless Sensor Network\\
3GPP & Third Generation Partership Project\\
\end{tabular}
\end{table}

